\documentclass{jsarticle}
\usepackage{fancybox}
\begin{document}
\title{DIC引き継ぎ資料}
\author{材料システム研究室某B4(当時)}
\date{\today}
\maketitle


\section{はじめに}
このファイルは、\date{\today}現在の情報に基づいて作成されています。日付が大きく乖離している場合にはこれが最新版かしっかりと確認をしたうえで読み進めてください。

DICを始めた頃は「めんどくさそう」とか「むずかしそう」とか「こんなんやるためにここきたんじゃねえ」\footnote{本職除く}とか思うかもしれませんが、とりあえず「便利」で「簡単」で「やる価値はある」\footnote{当社調べ}ものなのでマスターしてみてください。


\section{DICについて}
有川先生からDICについての資料を頂いていると思うので、原理とか使い方とか応用例とかは割愛させて頂きます。


\section{機材について}
2025年の代はDICが本職の方がいませんので、来年に期待。
\subsection{「異材接合界面の動的破壊靭性に関する研究」--内藤さん}
\begin{itemize}
  \item ひずみゲージをトリガーにした高精度カメラ\footnote{いいなあ~}
  \item DIC解析アプリ(改良DIC\_2021)
  \item あとは知らん
\end{itemize}

\subsection{「音波で収縮変形する構造に関する研究」--赤沢}
\begin{itemize}
  \item SONY製クセ強カメラ(持ち運びできるやつ)\footnote{SONY DSC-RX10M4}
  \item SONY謹製ファイル取り込みアプリ\footnote{どらごんPCにインストール済み}
  \item DIC解析アプリ(改良DIC\_2021)\footnote{この研究室のパソコン開けばこれに当たる}
  \item GraphR(CSVファイルからカラーマップ作ってくれるやつ)\footnote{ググれば出てくるしどらごんPCにも入ってるよ}
  \item 動画から8bitのbmp画像を指定秒数分作成してくれるPythonコード\footnote{どらごんPCなら実行可能・あとは知らん}
\end{itemize}


\section{解析手順}
\subsection{試験片作成}
\begin{enumerate}
  \item 試験片を白いペンで塗りつぶします。
  \item 外に出ます。\footnote{スプレーは外でやる!}
  \item 黒のスプレーを吹き付けます。
  \item 乾燥させて完成。
\end{enumerate}

\subsection{実験}
\begin{enumerate}
  \item 試験片を実験装置に取り付けます。このとき、膜がスピーカーとは反対側に向くようにしましょう。
  \item 共振周波数を探します。\footnote{絶対に変わってます。}
  \item 共振周波数を探せたら、LEDを点灯させて、音を流してSONY製クセ強カメラのHFRモードで記録します。
  \item どらごんPCとUSBケーブルで接続します。
  \item SONY謹製ファイル取り込みアプリで動画ファイルを吸い出してPythonコードに投げ入れましょう。\footnote{動画はアプリで吸い出さないとダメそう}
\end{enumerate}

\subsection{Pythonコードでbmp画像に変換}
詳しくはDICフォルダ内に入っている取扱説明書.txtを読んでください。
\begin{enumerate}
  \item Movieフォルダ内に動画を格納してください。\footnote{ファイル名は変更することを強く推奨(日付がおすすめ)}
  \item Pythonフォルダ内のPythonコード(extract\_frames.py)を実行してください。
  \item Resultフォルダ内の動画ファイル名のフォルダ内にbmp画像が生成されていることを確認してください。
\end{enumerate}

\subsection{DIC解析}
有川先生から頂いたDICについての資料をよく読んで操作してください。
\begin{enumerate}
  \item 撮影した動画を確認して何秒頃から音が流れ始めているのかを確認してください。\footnote{もちろん、動画ファイル内に音声なんて入ってません。スピーカーの振動から感じるのがおすすめ。}
  \item 秒数が確認できたら、秒数に24を掛けてください。\footnote{多分この動画は24FPS(Frames Per Second)撮影だったはず}
  \item 0000.bmpと「上記の答え」.bmp付近の画像でDIC解析を行ってください。
  \item 改良DIC\_2021フォルダ内のDebugフォルダにCSVファイル(Result.csv)が出力されているので、どこか別のところにコピーしてください。\footnote{コピーしなくてもいいけど上書きされるよ}
\end{enumerate}

\subsection{グラフ(カラーマップ化)}
出力されたCSVファイルをGraphRに投げ込んでください。まるでインスタ映え\footnote{しねえよ}しそうなカラーマップが出てきます。

\subsection{注意事項}
\subsubsection{こんなものを読んでいる人へ}
音波の実験テーマの人が書いていますので、音波以外のテーマの方はその研究テーマの先輩なり先生に相談してください。\footnote{DICに足を突っ込まなくていい人は突っ込まない方がいいです。}\footnote{音波組の人もわからなかったら(わからなくなる前に)先輩や先生に遠慮せずに聞いて下さい(最初はわからなくて当たり前です)。}

\subsubsection{試験片作成--スプレーの吹き付け}
直接試験片に対してスプレーを吹き付けてはいけません。ランダムパターンではなく単一の装飾になります。コツとしては、スプレーを上向きにして何回かスプレーを吹き付けてください。イメージとしては、上に噴射して落ちてくるスプレーのインクの粒を試験片に載せていく感じです。\footnote{何回かは先輩とかに教えてもらったほうがいいかも}

\subsubsection{試験片作成--下地について}
音波のDICを行う際に、下地に使用する白いマジックの塗膜の厚さによる振動の阻害によって、DICではない実験とは異なる振動を起こすことがあります\footnote{起こします!}。そのときは、下地を塗らずにPLAのまま黒のスプレーを吹いてください。これについては、2025/06/04に実施した実験において、収縮が継続することを確認済みです。

\subsubsection{実験--撮影時の照明について}
撮影時における照明は必要不可欠なものですが、この照明がレーザ変位計に悪影響を与えているのではないかという報告がいくつか\footnote{1つ}報告されています。

具体的には、照明を点灯させる前までは「収縮の挙動の継続」が確認できていたのにも関わらず、ライトを点灯させた途端に「収縮の挙動の継続をやめてしまった」うえに「拡張側に振れ始める」という\underline{造反的}変位を見せたものです。原因等究明とともに対応策を検討中です。

\subsubsection{実験--SONY製クセ強カメラのF値の設定}
このカメラは玄人向けのカメラなので普通のカメラのように一筋縄では行きません。以下の手順を経る必要があります。F値を大きくすることでピントが合う範囲を大きくすることができるようになる反面、得られる画像は暗くなります。\footnote{だからLEDを点灯させるんやで}
\begin{itemize}
  \item HFRモードに切り替える(上のダイヤルくるくる)
  \item (Fn)ボタンを探し出して押す
  \item 設定をM(マニュアル設定)に切り替え
  \item レンズの1番カメラ側のスライダを回してF値の調整
\end{itemize}

\subsubsection{実験--SONY製クセ強カメラのHFRモードについて}
このカメラは玄人向けのカメラ(再掲)なので普通のカメラのように素人がわかりやすいようになんて設計されていません。HFRモードで撮影を開始したら「取り込み中」と出ますが、このときに録画をしています。\footnote{もうちょいわかりやすい日本語ないん?}その後は、SDカードに書き込みのフェーズですので気をつけてください。このカメラのHFRモードは3秒間\footnote{こら、そこ。短っなんて言わない。}くらいしか録画できません。

\subsubsection{Pythonコードでbmp画像に変換--環境設定系}
研究室の中をクルクルと見回して"あっ、この人パソコンわかりそうだなぁ~"って人に聞いて下さい。\footnote{環境設定が趣味の人がいるかも知れません。}

\subsubsection{DIC解析--解析画像の設定等について}
この注意事項は音波組(SONY DSC-RX10M4を使用する人)のみ、影響が考えられます。DIC解析アプリ(改良DIC\_2021)は「bit深さが8bitのbmp画像のみ」が解析に使用できるそうです。上記のPythonコード(extract\_frames.py)を使用中の方は大きな影響はないと考えられますが、「ウイルス等の心配がある」や「環境設定に匙を投げた」等の理由で他人が使ったコードなんて使いたくねーよと思っている方はご注意をお願いします。

ちなみにbmp画像のbit深さは以下の手順でご確認ください。
\begin{enumerate}
  \item 「bit数を確認したいファイルを選択」--「右クリック」
  \item 「プロパティ(Alt+Enter)」
  \item 「詳細」--「イメージ」--「「ビット深さ」の数値を確認\footnote{8ならおけ~}」
\end{enumerate}

\subsubsection{DIC解析--解析時の範囲設定(Tips)}
範囲設定のpixel数がわかんねえよって思った人は生成されたbmpファイルを「ペイント」\footnote{フォトじゃねえぞ}で開いてみましょう。ポインタの位置のピクセル数が左下の方に出てきます。

\subsubsection{DIC解析--解析時の範囲設定}
膜の動きを見たいときに解析範囲を膜部分のみにしてしまうと解析時の誤差が乗っているのか、変位なのかがわからなくなります。そんなときには、解析範囲に少し背景部分を入れてあげると良いです。背景部分が誤差が大きくなり、膜部分に誤差が乗らない(緑色に近くなる)ようになれば成功と言えるでしょう。\footnote{これに気がつくまで3週間位費やした}

\subsubsection{DIC解析--枠のDIC解析について}
画像の探索範囲に枠部分を取り入れることで枠部の動きのDIC解析ができるのでは?という淡い期待を持って解析を行いましたが、メモリ関係のエラー(外部例外 EEFFACE)が出てきます。\footnote{枠が細すぎるから?}多分このエラーは1つの解析範囲を小さくしたのにも関わらず出現するので、プログラム内でクソデカ画像を扱おうとしていることから、確保したメモリ容量が足りていないことが原因で出てくると考えられています。\footnote{プログラムを作った張本人ではないので詳しいことはわかりません。}

\subsubsection{グラフ(カラーマップ化)--ある1点を中心とした減算について}
DIC解析は、その特性上カメラと試験片の固定台の\underline{相対的な}変位を見ることができるという計測方法に過ぎません。つまり、この解析においてカメラが「不動体」であればそれで良いのですが、そうは問屋が卸しません。また、カメラと試験片の固定台の振動が一致していればいいですが、まぁ~人生そんなにうまくはできていないんですね、これが。

そこで、活躍するだろうと思われて作成したのがCSVファイルの減算ツールです。このツールを使用することによって、「本来変位が出てはいけないはずの「固定端」」に変位が出てしまい、いかにも2次モードや3次モードに誤認させてくるカラーマップを退治することができます。

もちろん、皆さんこのExcelを開発したときの開発秘話だとか苦労した所、沼にハマった場所などを聞きたいですよね?

え?聞きたくない?(このセクションの執筆は深夜テンションでしているので(実話・2025/06/08 2:04)、周りから聞きたくないという幻聴が聞こえたので割愛します\footnote{あー話したかったなあ}。)

ちなみにこのカラーマップをスライドにデカデカと張ったうえで、研究報告会で「膜の振動モードは2次モードか3次モードと考えられるので、八角形枠は収縮が継続するのです。(ドヤァ)」と言ったところ、A先生に「これ、1次モードだし、2次モードか3次モードになったところで収縮継続の理由にはならない」と言われ、無事爆散したのはこの私です。

\subsubsection{グラフ(カラーマップ化)--減算CSVファイルのGraphRでの取り扱い}
減算を始めた背景について理解してもらったところで、その「減算したCSVファイル」をカラーマップ化すると思います。これまた、罠があるのです\footnote{人生罠だらけ}。なにも設定を変更せずにGraphRに減算したCSVファイルを投げ入れると、減算前のCSVファイルのカラーマップと同じカラーマップが出てきます。

なんと、このアプリ頭が良過ぎることから人間を小手先で騙そうとして、カラーバーの縮尺のみを変えるとかいうよくわからない暴挙に出ていました。皆さんは頭が良いのでこんなことすぐに気がつくと思うのですが、私は無理でした。「何も変わってないやん」とExcel様に疑心暗鬼になり、1日経ってからよく見たら縮尺が変わっていました。

減算したら縮尺を変えてあげてください\footnote{多分、めっちゃきれいな1次モードが見れます。}。

\subsubsection{グラフ(カラーマップ化)--CSVファイルの取り扱い}
Debugフォルダ内に生成されたCSVファイルは、未加工の状態ではX軸(横向き)のpixelの変位しか見ることができません。Y軸(縦向き)のpixelの変位を見たくなったら、CSVを開いて4列目のデータを3列目にコピペする必要があります。\footnote{これもCSVファイルを複製してからやること推奨}

\subsubsection{グラフ(カラーマップ化)--軸の取り方}
X軸は右方向が正の方向・Y軸は下方向が正の方向


\section{さいごに}
DICというツールは比較的静的な変位を起こしつつ、小さな変位を起こす研究テーマ\footnote{この研究室なら亀裂とSMP}に最適だと考えられます。音波は結構動的な研究ですので少し相性は悪いかもしれませんが、諦めずに頑張ってみてください。マスターすれば結構面白いかもしれません。

よくわからないところや修正希望、項目の追加希望等ありましたら研究室内で叫ぶとか、ホワイトボードになんかメモっとくとかしてください。それ相応の対応はさせて頂きます。こういうことを書いている人はフィードバックに関しては特に怒ったりはしません。どんどんご意見お待ちしております。
\quad
\begin{flushright}
  \date{\today}\ \ 材料システム研究室某B4(当時)
\end{flushright}
\end{document}