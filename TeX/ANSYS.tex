\documentclass{jsarticle}
\usepackage[hyphens]{url}
\usepackage{ascmac}
\usepackage{fancybox}
\usepackage[dvipdfmx]{graphicx}
\begin{document}
\title{ANSYSがトラブったときの対応方法まとめ}
\author{材料システム研究室某B4(当時)}
\date{\today}
\maketitle


\section{はじめに}
このファイルは、\date{\today}現在の情報に基づいて作成されています。日付が大きく乖離している場合にはこれが最新版かしっかりと確認をしたうえで読み進めてください。

このPDFファイルだか紙媒体に目を通しているということは、ANSYSがストライキを起こしたことによって研究が止まってしまうのではないかと心配していることとご拝察いたします。

このPDFファイルだか紙媒体ではこれまでに確認された、ANSYSのストライキの例とそれに果敢に立ち向かい解決したときの対応が綴られていくはずです。しかし、筆者の心持ちが変わったりすればこのように行くとは限りません。

本来、この文章はANSYSのストライキを鎮圧させる方法について書かれたものなので、材料システム研究室のNASの中でずっと眠っておくべきです。ANSYS等のアプリケーションは「何も異常がない状態(ただ、条件を投げれば計算をしてくれる)」が「普通」なのであって、このようなファイルがありがたがられていることや活用されているということは「普通」ではないと断言できるでしょう\footnote{ねっ?ANSYSのエンジニアさん}。

と、大きな口を叩いていますがこのファイルを作成していて、このPDFファイルだか紙媒体の気持ちになったときに、ずっとNASで保管され続けた\footnote{ANSYSがある限りそれはないと思うが…}挙げ句に誰にも知られずにひっそりと捨てられることを思ったら涙がちょちょ切れた\footnote{な訳あるか}ので、Tipsも拡充していくと思われます。

さぞかし、不安でしょう。こんなときは焦らず深呼吸をしながらANSYSに立ち向かいましょう。焦ったまま操作すると取り返しのつかない操作をしてしまうかもしれませんよ?

しかしながら、このPDFファイルだか紙媒体がANSYSのストライキをやめさせる一助となれば幸いです。


\newpage

\section{ストライキの例と対応}
\subsection{なんかライセンスエラー(2025/03/21頃ー共用PC)}
\subsubsection{ストライキの例}
このときのライセンスエラーはパソコンのメモリを増設したとかでエラーが発生。

エラーの内容的には「お前のパソコンライセンスが通ってないからXX分後にANSYS落とすからな。保存くらいしとけよ。」とかいうストライキ中とは思えないほどの優しさを見せてくれたエラー。
\subsubsection{ストライキへの対応}
ANSYSを削除したあとにExcelファイル(ANSYSインストール.xlsx)の指南に従ってインストール。
\begin{enumerate}
  \item 「検索」--「Workbench」--「↓」--「場所を開く」
  \item 20XX(23) R1のフォルダごと削除
  \item Excelの手順の通りにインストール
  \item インストール時にDiscoveryのチェックは外す\footnote{チェックを入れていても大丈夫だけど、よくわからない人はチェックを外す。}
\end{enumerate}


\subsection{なんかライセンスエラーその2(2025/03/27頃ー土田さんPC)}
\subsubsection{ストライキの例}
急にライセンスが不安定になり、エラーが出るようになる。これぞ、ANSYSストライキの真骨頂。

これもまた、ご丁寧にも「お前のパソコンライセンスが通ってないからXX分後にANSYS落とすからな。保存くらいしとけよ。」とかいうストライキ中とは思えないほどの優しさを見せてくれたエラー。
\subsubsection{ストライキへの対応}
ANSYSを削除したあとにExcelファイル(ANSYSインストール.xlsx)の指南に従ってインストール。前回とは削除の方法を変えてみた。
\begin{enumerate}
  \item プロジェクトを保存\footnote{これだけはやれ。これやらずにプロジェクト吹っ飛ばしたらANSYSのストライキ対応どころではない。}
  \item 「検索」--「ANSYS 20XX~」的に入力
  \item 「Uninstall ANSYS 20XX R1」を管理者として実行
  \item 削除が終わったらパソコンを再起動\footnote{無意味だなあと思ってもやれ。}
  \item Discoveryのチェックは外すところで「必要なストレージ容量」が0MBになっていないことを確認。\footnote{そのままインストールしたら、アホみたいに早く終わるけどただの時間の無駄だからしっかり確認を。}
  \item Excelの手順の通りにインストール
  \item インストール時にDiscoveryのチェックは外す
  \item インストール後は何回かWorkbenchを立ち上げてライセンスが通っているかを確認\footnote{今回は2回}\footnote{モーダルとかの解析項目が出現すればストライキ鎮圧成功}
\end{enumerate}


\subsection{ANSYSに関する相談(2025/04/28頃ー共用PC)}
\subsubsection{相談内容}
今までは、エクスプローラー上でagdbファイル(ANSYSのモデリングファイル)をダブルクリックすれば、Workbenchが開いていたとのこと。今はできない\footnote{再インストールのときにやらかした可能性大}。
\subsubsection{対応策}
Workbenchを開いた状態でファイルをドラッグアンドドロップ。新規のプロジェクトにしてファイルから開くでも行けた。これで解決。

2025/06/03に再インストールをしたところ、エクスプローラでagdbファイルをダブルクリックすれば開けるようになりました。この手のエラーは再インストールをしてあげてください。


\subsection{ライセンスで恐怖を煽ってくる系エラー(2025/04/29頃ー土田さんPC)}
\subsubsection{ストライキの例}
急にライセンスが不安定になり、エラーが出るようになる。これぞ、ANSYSストライキの真骨頂その2。

これは、「お前のパソコンライセンスが通ってないからXX日後にライセンス切れるからな。覚悟しとけよ。」とかいう、いかにもストライキ中のエラー。
\subsubsection{ストライキへの対応}
エラーが出たり出なかったり。サーバへの接続がうまく行けばエラーは出ないのかもしれない。エラーが出たらhostsファイルの書き換えを行う予定。

保護経過処分
\begin{enumerate}
  \item 保護経過観察中
  \item 誠心誠意対応中
\end{enumerate}


\subsection{ライセンスで恐怖を煽ってくる系エラー(2025/04/29頃ー共用PC)}
\subsubsection{ストライキの例}
急にライセンスが不安定になり、エラーが出るようになる\footnote{一昔前のWindowsかよ}。これぞ、ANSYSストライキの真骨頂その2-1。

これは、「お前のパソコンライセンスが通ってないからXX日後にライセンス切れるからな。覚悟しとけよ。」とかいう、いかにもストライキ中のエラー。
\subsubsection{ストライキへの対応}
エラーが出たり出なかったり。サーバへの接続がうまく行けばエラーは出ないのかもしれない。エラーが出たらhostsファイルの書き換えを行う予定。
\begin{enumerate}
  \item エラーが出る。余命宣告2025/05/13の夜\footnote{もうちょい安定したサーバ運営をやれ}
  \item 2025/05/14確認時、エラーが出ない。延命治療された様子\footnote{いつ落ちるかな?}。
  \item 2025/05/15現在、ANSYSさんは延命治療が失敗され、お亡くなりになられました。Mechanicalのライセンスがないようです。
  \item 上記の数十分後、再度蘇生されております\footnote{どんだけしぶとい命やねん}。
\end{enumerate}

2025/06/03付で再インストールが完了していることから、この件につきましては「終結宣言」を宣言いたします\footnote{「皆様、お手を拝借。いよ~っ!」「 た・た・たん  た・た・たん  た・た・たん  たん!」「ありがとうございました!」}。


\subsection{新規インストール後にライセンスが通らないエラー(2025/04/30頃ー平野さんPC)}
\subsubsection{ストライキの例}
新規インストール後にライセンスが通らない\footnote{いつものことかと軽視してたら大きめなストライキだった}。

ANSYSの起動が非常に遅い上に、サーバのタイムアウトとかいうエラー。有川研のネットワークが悪いと考えられる。

井上さんPCにANSYSを新規インストールしたところ、同様の事案が発生。CPUとかの世代とANSYSの相性の可能性がありそう\footnote{この世代のIntelのCPUはCPUのせいにしとけばなんとかなる。}。
\subsubsection{ストライキへの対応}
ファイヤーウォール解除してもダメ、Excel入れてもダメ、再インストールしてもダメ。有川研のストライキ鎮圧の最終奥義である、再インストールを試みるも解決には至らなかった。

2025/05/14に来て頂いた先生(石田先生)によれば、ライセンスサーバは絶賛稼働中とのことであり、ネットワークの問題説が濃厚。

Cドライブの奥深くにあるhostsファイルを書き換える。このファイルのファイルパスは(C:/Windows/System32/drivers/etc/hosts)となっている。
\begin{enumerate}
  \item ファイヤーウォールで送信・受信でポート番号1055/2325をTCPで開放→ダメ
  \item Excel関係のエラーが出ていたのでExcelインストール後にWorkbench起動→ダメ
  \item Uninstallアプリで再インストール→ダメ
  \item pingコマンドで3つのサーバに応答確認→kyodo-ad以外はタイムアウト
  \item hostsファイルの書き換えによって無事にストライキ鎮圧
\end{enumerate}

\paragraph{hostsファイルの書き換えについて}\quad\par
hostsファイルの書き換えは簡単ですが、しっかりとコピーを取ってからやってくださいね。不慣れな場合は「ほいほい」と進まないこと。しっかり確認したうえで行ってください。こんなんでパソコンがバグりまくったら、ANSYSストライキの鎮圧どころではなく、めんどくさいからね\footnote{言ったからな}。
\begin{enumerate}
  \item hostsファイルがあるパス(C:/Windows/System32/drivers/etc/hosts)に行く。
  \item もともとあるhostsファイルをコピーして別の名前にリネームする\footnote{hosts\_oldとかにした}。
  \item hostsファイル\footnote{hosts\_oldじゃねえぞ}を「右クリック」--「メモ帳で編集」
  \item 一番最後の部分に以下の3行を付け加える\footnote{見栄えの関係上1行で済ませていますが、「/」が「改行」だと思ってください。}。
  \item 133.26.87.248 ansys-server-3/133.26.148.40 mech/133.26.86.73 kyodo-ad\footnote{1行だけど3行にするんだからな}\footnote{構成:IPアドレス[半角スペース]ホスト名}
  \item 保存すれば設定完了\footnote{権限はあると思うけど管理者権限必須}
\end{enumerate}\par
この手順を見て少しでも「無理ぽ~」って思った人は遠慮せずに平野さんPCに直行。hostsファイルがあるパス(C:/Windows/System32/drivers/etc/)に入ってhostsファイルをコピーして自分のPCの同じパスに入れましょう\footnote{これが簡単で最速です}。

間違えてhostsファイルを消してしまった、そんな人に朗報です。この文書の作者はこんなこともあろうかと以下にhostsファイルの内容を書き記しています\footnote{頑張ったものは報われる世の中じゃないとね}。

このときの「従来のhostsファイルの内容」はあくまでなにも書き換えをしていない、かつ今回の環境においての「内容」です。つまり、使用環境が異なればhostsファイルの内容も異なる可能性があるということになります。実際にこのファイルの作成者のノートパソコンでのhostsファイルは異なる内容\footnote{\#が付いている部分は同一}でした。そんなときは、元にあったものの下に改行をして付け足せばいいと思います\footnote{知らんけど}。

\begin{itembox}[l]{従来のhostsファイルの内容}
  \#\ Copyright (c) 1993-2009 Microsoft Corp.\par
  \#\par
  \#\ This is a sample HOSTS file used by Microsoft TCP/IP for Windows.\par
  \#\par
  \#\ This file contains the mappings of IP addresses to host names. Each\par
  \#\ entry should be kept on an individual line. The IP address should\par
  \#\ be placed in the first column followed by the corresponding host name.\par
  \#\ The IP address and the host name should be separated by at least one\par
  \#\ space.\par
  \#\par
  \#\ Additionally, comments (such as these) may be inserted on individual\par
  \#\ lines or following the machine name denoted by a '\#' symbol.\par
  \#\par
  \#\ For example:\par
  \#\par
  \#\ \ \ \ \ \ 102.54.94.97\ \ \ \ \ rhino.acme.com\ \ \ \ \ \ \ \ \ \ \# source server\par
  \#\ \ \ \ \ \ \ 38.25.63.10\ \ \ \ \ x.acme.com\ \ \ \ \ \ \ \ \ \ \ \ \ \ \# x client host\par
  \quad\par
  \#\ localhost name resolution is handled within DNS itself.\par
  \#\ \ \ \ \ 127.0.0.1\ \ \ \ \ \ \ localhost\par
  \#\ \ \ \ \ ::1\ \ \ \ \ \ \ \ \ \ \ \ \ localhost\par
\end{itembox}

\begin{itembox}[l]{変更後のhostsファイルの内容}
  \#\ Copyright (c) 1993-2009 Microsoft Corp.\par
  \#\par
  \#\ This is a sample HOSTS file used by Microsoft TCP/IP for Windows.\par
  \#\par
  \#\ This file contains the mappings of IP addresses to host names. Each\par
  \#\ entry should be kept on an individual line. The IP address should\par
  \#\ be placed in the first column followed by the corresponding host name.\par
  \#\ The IP address and the host name should be separated by at least one\par
  \#\ space.\par
  \#\par
  \#\ Additionally, comments (such as these) may be inserted on individual\par
  \#\ lines or following the machine name denoted by a '\#' symbol.\par
  \#\par
  \#\ For example:\par
  \#\par
  \#\ \ \ \ \ \ 102.54.94.97\ \ \ \ \ rhino.acme.com\ \ \ \ \ \ \ \ \ \ \# source server\par
  \#\ \ \ \ \ \ \ 38.25.63.10\ \ \ \ \ x.acme.com\ \ \ \ \ \ \ \ \ \ \ \ \ \ \# x client host\par
  \quad\par
  \#\ localhost name resolution is handled within DNS itself.\par
  \#\ \ \ \ \ 127.0.0.1\ \ \ \ \ \ \ localhost\par
  \#\ \ \ \ \ ::1\ \ \ \ \ \ \ \ \ \ \ \ \ localhost\par
  133.26.87.248\ ansys-server-3\par
  133.26.148.40\ mech\par
  133.26.86.73\ kyodo-ad\par
\end{itembox}

\begin{boxnote}
  技術的Tips:

  hostsファイルはDNSサーバにIPアドレスとホスト名の読み替えを要求する前に読み込むファイルらしいです。つまり、うまく設定してやれば、あるサイトだけ……おっと誰か来たようだ。
\end{boxnote}


\subsection{マテリアル設定ができないエラー(2025/05/05頃ー土田さんPC)}
\subsubsection{ストライキの例}
急にマテリアル関係の設定ができないようになる。これぞ、ANSYSストライキの真骨頂その3\footnote{いい加減にせい}。

これはなんと、エラーメッセージすら出さないというアプリケーションとは思えないほどのゴミ加減を見せてくれたエラー\footnote{情報処理実習の授業でやったよね?「return -1」。これすらできないアプリケーションに研究人生託してます。}。
\subsubsection{ストライキへの対応}
ANSYSを削除したあとにExcelファイル(ANSYSインストール.xlsx)の指南に従ってインストールしたらいつも通り直ったんですよ…と言いたいところですが、直りませんでした。

2025/05/14に来て頂いた先生(石田先生)によれば、ANSYS界隈で初の事例らしいです。
\begin{enumerate}
  \item 「表示(V)」--「ウィンドウレイアウトをリセット」
\end{enumerate}

たったこれだけで修正完了。時間を返せ\footnote{まぁ、こんなときもあります。研究も一緒です。「時間を返せ」と思った研究もいつかは役に立つ日が来るかもしれません。明けない夜などありません。}。


\subsection{マテリアル設定ができないエラー(2025/06/03頃ー共用PC)}
\subsubsection{ストライキの例}
Workbench起動時に「ルート要素が見つかりません」\footnote{さんはい、皆さんご一緒に「そんなもん自分で探せ」}というエラーが出てくるストライキ。この手のエラーは前々から出てはいたのですが、誤魔化し誤魔化し使い続けてたツケが回って来ましたね。皆さん、後回しは良くないですよ。

これについてはANSYSさんは学習したのか、「エラー」は出してくれましたが、エラー内容が意味不明でした\footnote{情報処理実習の授業でやったよね?「return -1」に付随したエラーメッセージの付け方。これすらできない(以下略)}。

\subsubsection{ストライキへの対応}
必殺奥義\footnote{この手を使いすぎてもはや必殺ではない?}というかなんというか再インストール。

今回は、Uninstall用のアプリケーションが星の彼方に行ってしまったので\footnote{存在しなかったんだよ}、手作業で削除しました。
\begin{enumerate}
  \item C--ProgramFiles--ANSYS Incのフォルダごと削除
  \item 検索--Workbench2023--ファイルの場所を開く→このフォルダごと削除
  \item いつも通り再インストール
  \item 機能を選択する所では「Discovery」以外はチェックを入れた状態
  \item CADジオメトリは新基準 AutoCAD・Fusion・SolidWorksに対応
\end{enumerate}

今回は一発でライセンスが通りました。素晴らしすぎます。これは、ANSYSのお陰では\underline{決して}なく、ANSYSを使って\underline{やっている}皆さんの日頃の行いが良いからに決まっています。


\subsection{ライセンスが5n分後に切れるよ系エラー(2025/06/11頃ー内藤さんPC)}
\subsubsection{ストライキの例}
ライセンス切れる系のエラーが再来しました。しかも、急に(内藤さん談)。このときのエラーはライセンスサーバが見つからない系のエラーでした\footnote{皆さん、もうどの手段を使うのかわかりますよね?再インストール?お帰りください。}。5n分後(nは12未満の自然数だったはず…)にライセンスが切れるからなと「ご丁寧にも」教えてくださったANSYSさんに感謝を申し上げつつ、ササッと修理\footnote{おい、ANSYSあの頃の材料システム研究室ちゃうぞ。}。

\subsubsection{ストライキへの対応}
もちろん\underline{あの頃の}材料システム研究室ではないので、hostsファイルの書き換えという必殺技があります。前述の通りhostsファイルのあるパス(C:/Windows/System32/drivers/etc/)に書き換え後のhostsファイルをぶち込んで、Workbenchの再起動をしたところエラーが出ませんでした。


\subsection{メカニカルデータベースがないよ系エラー(2025/06/13頃ー共用PC)}
\subsubsection{ストライキの例}
エラーを起こしたフォルダは、平野さんPCと共有している状況。

解析を回そうとしたら、メカニカルデータベースがないとかほざきだす始末。原因としてはメカニカルデータベース(参考:[当該ファイルパス](今開いているプロジェクト名)\_files/dp0/global/MECH/SYS(-X).mechdb)がない状態。今回はSYS-2.mechdbが星になっていました\footnote{消えていたってこと}。

\subsubsection{ストライキへの対応}
SYS-X.engdのコンパイル?後の形がmechdbになっている気がした\footnote{確証なんてないですよね}ので(ファイルサイズ一緒だったし、行数も一緒だったし、内容も一緒ぽかったし)、SYS-1.mechdbをコピーしたうえでSYS-2.mechdbにリネーム。

こんだけで直りましたよ。過去一簡単。最高すぎる。

とうぬぼれていたのもつかの間、mechdbファイルはメカニカルデータ(材料データ)のみを管理しているのかと思ったら、解析の種類から何やらを管理しているものでした。

engdファイルからmechdbファイルを錬金術で生成できないか考えたうえで、engdファイルをプロジェクトで開けるのならば行けるかもしれないと考えましたが、錬金術は所詮錬金術でした\footnote{鉛から金は作れません。}。ちなみに、engdファイルはメカニカルデータ(材料データ)のみを管理しているファイルでした。

ネットの海は、「mechdbファイルをmechdatファイルに拡張子変更すればプロジェクトで開けるよ。」とか言ってますが、そもそもそのファイルがないねん。なあ。さらに、mechdatファイルはプロジェクトで開けません\footnote{ネットの海は信用できませんね。}。

しかし、なぜこうなったかは\underline{もちろん}不明です\footnote{再発の可能性あります}。

一応、考えられる原因としては転送中や書き込み中にWorkbenchを消した可能性があるとのことですが、まるでこちら側に全責任があるかのような発言がなされたことは誠に遺憾であります\footnote{遺憾砲発射!!}。


\subsection{レイアウトがおかしいよ系エラーその1(2025/06/24頃ー平野さんPC)}
\subsubsection{ストライキの例}
プロジェクト画面で発生。今までは、解析種類が確認できていましたが、なぜかできなくなりました。

エラーはしっかりと吐きました\footnote{ANSYSならこれだけで褒められる。よくやった。}。解析種類のブロックと線図ではなく、白い背景に黒文字で「現在の選択で適用可能なツールボックスアイテムはありません。」というエラー。

ツールボックスを最小化したら出るエラーっぽい。全画面にこのエラーが出るということは多分、概念図も最小化してしまっている。

\subsubsection{ストライキへの対応}
レイアウト関係のエラーだったので、「マテリアル設定ができないエラー(2025/05/05頃--土田さんPC)」の対応に従って対応。
\begin{enumerate}
  \item 「表示(V)」--「ウィンドウレイアウトをリセット」
\end{enumerate}

これだけで本当に直りましたよ。ANSYS修理人生の中で初めてエラー内容とエラーメッセージが対応していた気がする\footnote{偉いぞ~。ANSYS君。}。


\subsection{レイアウトがおかしいよ系エラーその2(2025/06/24頃ー平野さんPC)}
\subsubsection{ストライキの例}
Mechanicalで発生。今までは、アウトライン(左上)・詳細情報(左下)・3Dモデル(中央上)・グラフ(中央左下)・テーブルデータ(中央右下)が配置されていましたが、詳細情報(左下)がなぜか星になりました\footnote{消えていたってこと}。

ANSYSって基本的にCtrl+Zとかの「戻る」コマンドが使用できずに最初の方は結構消して「あっ」となることがチラホラ。

\subsubsection{ストライキへの対応}
\paragraph{初期対応}\quad\par
レイアウト関係のトラブルなので、とりあえずリセットすればいいっしょくらいの心持ちで対応\footnote{研究もこんくらいの心持ちで}。
\begin{enumerate}
  \item 「上のメニュー」--「管理」--「レイアウトのリセット」
\end{enumerate}

これのボタンはHさんに見つけてもらいました。これを書いている人がアホなので\footnote{多分、にほんごが読めなかった}。

これで直った感は出ていたのですが、Hさんの彗眼からは逃れられませんでした\footnote{いくらANSYSでも}。グラフとテーブルデータの項目が消えました\footnote{これを書いている人は気が付きませんでした。多分、きおくりょくがない。}。

\paragraph{詳細対応}\quad\par
頑張って項目を探しまして、やっと発見しました。
\begin{enumerate}
  \item 「上のメニュー」--「管理」--「「グラフ」と「テーブルデータ」をONにする」
\end{enumerate}

\paragraph{通常のレイアウト設定について}\quad\par
(何故か……!!全然……!!分からないけどぉ!!たまたまぁ、偶然!!)設定がおかしくなったりとかいじったりしてしまったとき用に「通常「ON」になっている項目」を記しておきます。
\begin{itembox}[l]{通常「ON」になっている項目}
  \begin{enumerate}
    \item グラフィックツールバー
    \item ステータスバー
    \item アウトライン
    \item 詳細
    \item テーブルデータ
    \item グラフ
  \end{enumerate}
\end{itembox}

\paragraph{レイアウトのテンプレート化}\quad\par
ANSYSはレイアウトを散らかすプログラムが組まれているのかは分かりませんが\footnote{もはやウイルスだろ}、レイアウトをテンプレートにすることができるみたいです。
\begin{enumerate}
  \item 「上のメニュー」--「ユーザー定義」--「レイアウトを格納」--「名前を決める」
\end{enumerate}

あとは、レイアウトが狂ったときに呼び出すだけ。


\subsection{データが破損してしまったよエラー(2025/09/01頃ー共用PC)}
\subsubsection{ストライキの例}
今回の事例は、ANSYSのストライキとも言えずANSYSが悪いわけでもないのですが便宜上ここに分類。今回の悪玉はWindows Update\footnote{マルウェア}ですね。

何も求めていなければ、要求もしていないのに勝手にWindows Update\footnote{マルウェア}が走ったことによって、PCがシャットダウン。もちろん、解析もシャットダウン。ついでに、研究もシャットダ…誰かが来たようだ。

まぁ、解析が失敗したことはしょうがない。再解析するために時刻歴応答解析のタブを開こうとしたらエラーが出てきた。

データベースの破損で開けません。

あらあらあらと。ANSYSくん自分のエラーでなければしっかりとエラーメッセージが吐けるではないかと。しっかりとエラーメッセージには問題のあるファイルのパスが載っていました。mechdbファイルって言うんですけどね。

\subsubsection{ストライキへの対応}
ANSYSくんが最後の力を振り絞って出したメッセージに書かれたパスに行ったらしっかりmechdbファイルが存在していた。これは2025/06/13の事例とは異なるぞと。

そのとき、おぼろけながら浮かんできたんです。拡張子を「mechdb」から「mechdat」に変えるということを。

拡張子をmechdatに変えたあとWorkbenchでプロジェクトを新規で作成して、ドラッグアンドドロップをしたら開けたんですね。これが。

あとは特に保存もせずに閉じて、元のプロジェクトを開いたんですね。そしたら…なんと…開けてしまいました\footnote{こら、そこ。つまらないなんて言わない}。

データの復活はしたのですが、ファイルの中身の整合性は全く取れていませんでした。相当前のデータがロールバックしていました。

今回の事例では、ジオメトリや圧力の情報がロールバックしていて「わぁ、懐かしい(怒)」状態。感覚としては、この解析プロジェクトの初っ端に戻る感じかな。

あとは頑張って整合性取ってね。


\newpage


\section{ANSYSのインストール方法}
\subsection{ANSYSのインストール方法(川端さん著)}
川端さん(2024卒業)が非常に素晴らしいANSYSのインストール方法に関するExcelファイルを作ってくださったのですが、パソコンの新規導入時などにおいて、Microsoft謹製アプリケーション\footnote{Excel}がインストールされていない状況では、閲覧すらできないという人権問題に直結しそうな問題であることから、このファイルに連結・継承することといたしました。

\paragraph{ANSYSインストールの準備}
\begin{enumerate}
  \item プロジェクトを全て保存し、それを確認したうえでUninstall ANSYS 20XX R1を管理者権限で実行\footnote{ちゃんと確認しろよ。バグった存在価値\underline{が}ないANSYSと自分の成果である存在価値\underline{しか}ないプロジェクトが消え去るぞ。クソなアプリケーションと自分の「素晴らしい」プロジェクトが同格に扱われるってことだからな。所詮パソコンのくせに。}
  \item パソコンを再起動
  \item ANSYSインストール用のISOファイルが格納されているOneDriveに入る。
  
  (\url{https://meijiuniversity-my.sharepoint.com/:f:/g/personal/arikawa_meiji_ac_jp/EurBP0e-OK9AlqJ73WmQ3LcBV9DQ5-wTUyKm5MIbrtJHQQ?e=jy2Pwh})\footnote{URLのハイパーリンクが折り返しの前までしか踏めません。大変申し訳ございません。\doublebox{仕様です。}}
  \item URLが開けたらその時の最新版のフォルダを開く。
  \item Disk1~3.isoをダウンロード\footnote{3ファイルのダウンロード}
  \item Disk1~3.isoをそれぞれ「右クリック」--「プログラムから開く」--「エクスプローラー」
  \item Disk1.isoを開き、一番下にある「setup.exe」を「右クリック」--「管理者として実行」
\end{enumerate}

\paragraph{ANSYSのインストール(本編)}
\begin{enumerate}
  \item 同意事項を「同意(上)」にチェックして「次へ」\footnote{もちろん、しっかり読んで確認したうえで}
  \item インストールディレクトリの変更\footnote{変人以外はそのまま、そもそも他人が使う可能性があるPCでインストールディレクトリの変更なんてやるな}
  
  特に変更なし
  \item ライセンスサーバの設定
  
  \begin{itemize}
    \item Ansysライセンス相互接続ポート番号→2325(デフォルト\footnote{債務不履行のことではない})
    \item Ansysライセンス相互接続ポート番号→1055(デフォルト)
    \item ライセンスサーバの数の選択→3-サーバー(冗長\footnote{じょうちょうと読みます}3台構成)
    \begin{itemize}
      \item ホスト名(マスター):ansys-server-3.mind.meiji.ac.jp
      \item ホスト名2:mech.mind.meiji.ac.jp
      \item ホスト名3:kyodo-ad.mind.meiji.ac.jp
    \end{itemize}
  \end{itemize}

  \item インストールする製品(複数可)の選択\footnote{複数選べなくてどうすんねん}
  
  Discoveryのチェックを外す\footnote{まぁ、外さなくてもなんとかなりますよ、だからといってチェックを入れたままにしない}。
  \item CADジオメトリインターフェースの設定
  
  「Yes, 手動で選択します(上)」を選択して「次へ」
  \item 設定するCADジオメトリインターフェースの選択
  \begin{itemize}
    \item Auto CAD(一番上)
    \item Fusion 360(上から6番目)
    \item SOLIDWORKS(上から11番目・下から2番目)
  \end{itemize}
  \item 選択した項目用の設定対応の選択
  
  多分、2つ出てくると思うけど、両方下。「Workbench Associative Interface」を選択。「次へ」をクリック\footnote{相違なんて見なくていい}
  \item インストールの開始
  
  ほったらかしでいいと思った?まだ、やることが続きます。しばらくすると、2の場所を聞いてきます\footnote{自分で探せや。(パソコンに言っています)}。丁寧に回答して上げましょう。

  \begin{itemize}
    \item 「参照」--「(さっき開いた231-2.dvdが入っているであろうフォルダを探し出して)フォルダを選択」--「次へ」
  \end{itemize}

  しばらくすると、3の場所を聞いてきます\footnote{自分で探せや。(パソコンに言っています)}。丁寧に回答して上げましょう。

  \begin{itemize}
    \item 「参照」--「(さっき開いた231-3.dvdが入っているであろうフォルダを探し出して)フォルダを選択」--「次へ」
  \end{itemize}
  
  しばらくお待ち下さい。まもなく終了です。

  最後にアンケートがうんちゃらみたいなのが出てくる。答えたくなければチェックを外して「閉じる」\footnote{そら、答えたくないわな。こっちは苦しめられてるんだから。}
\end{enumerate}

\paragraph{イーサネットの設定}\quad\par
原文ではここから先の記述がありませんでした。ここでは、原文の継承を目的としていることから、原文の著者(川端さん)の意思を尊重し\footnote{何を書きたかったのかがわかりませんので}、この章は空白の章とさせて頂きます。

\subsection{ANSYSのインストール方法(追加)}
ここからの章ではANSYSの再インストールについて、川端さんの説明では至らない部分を補わさせて頂きます。

これは、勝手な想像ですが川端さんは「イーサネットの設定」の項目に優先するドメインの設定について書くつもりだったのだろうと思っています。これを設定することによって、サーバの設定の中で毎回記述していた、「mind.meiji.ac.jp」を打ち込む必要が「多分」なくなるからです\footnote{確証なんてない}。

Windows11から「そーゆー系の設定」\footnote{ニッチ系の設定}が非常に奥深くに行ってしまって設定する気にならなかったので、気が向いたらやってみてください\footnote{これは悪い意味でMacに憧れすぎ}\footnote{???「憧れるのをやめましょう」}。

「気が向いたらやってみてください」なんて言っていますが、NASを漁っていたらネットワークの設定のWordが出てきたことから、ファイルの作成者が「気が向いて」しまいました。
\begin{enumerate}
  \item 「コントロールパネル」--「ネットワークとインターネット」--「ネットワークと共有センター」--「イーサネット」--「プロパティ」--「インターネットプロトコルバージョン4(TCP/IPv4)」
  \item ウィンドウが開いたら「詳細設定(V)」
  \item 「DNS」タブを開く
  \item 「以下のDNSサフィックスを順に追加する」--「追加(D)」
  \item 「ドメインサフィックス」--「mind.meiji.ac.jp」--「追加」
\end{enumerate}

Wordファイルの中にIPアドレスとDNSサーバのアドレスについての設定が書いてあったので追加収録しておきます。元からIPアドレスとDNSサーバの設定があるようでしたら変更の必要はないです\footnote{多分、「自動的に取得する」になってると思います。もしそうなら変更する価値あり}。

\begin{itembox}[l]{「次のIPアドレスを使う(S)」編}
  \begin{itemize}
    \item IPアドレス(I):192.168.1.6
    \item サブネットマスク(U):255.255.255.0
    \item デフォルトゲートウェイ(D):192.168.1.1
  \end{itemize}
\end{itembox}

\begin{itembox}[l]{「次のDNSサーバのアドレスを使う(E)」編}
  \begin{itemize}
    \item 優先DNSサーバー(P):192.168.1.1
    \item 代替DNSサーバー(A):設定なし
  \end{itemize}
\end{itembox}


\newpage


\section{ANSYSの引き継ぎ(成田さん伝)}
2025/03/14かな?成田さんというANSYSマスターの偉人に来て頂き、ANSYSについて色々教えて下さいました\footnote{これを書いている本人は当日、鹿児島か熊本にいました。成田さん、申し訳ありません。}。このときの「教え」をメモしてくれたSくん(B4(当時))のお陰で今、まとめられています。感謝しかありません\footnote{だって本人は九州を旅していたのですから。}。そのメモを元にリマスターしたものがこの文章です。
\subsection{プロジェクトの最初}
エンジニアリングデータは材料の設定を示します。また、ジオメトリはDesign Modelerを使用します。

\subsection{メッシュについて}
基本的には立方体が良い。しかし、破壊の解析を行いたい場合には三角錐がオススメ。

\subsection{ステップ数について}
ステップ数とは、解析の区画を表す。このとき、「時間自動ステップ」はONにする。また、定義方法は「サブステップ」。

サブステップとは、1秒間に計算する回数のこと。このとき、最小は「どうでもいいところ」、最大は「大事なところ」に適用するように使い分ける。

解析時間を1/2にしたときに、サブステップも1/2にしないと計算条件が統一されないという問題が発生する。

ステップ数を分けて、どうでもいいところのサブステップを小さくするなどのテクニックもある。

※以下最高に謎

2ステップ目、28000、28000、10の8乗とか\footnote{???}。

\subsection{変形の設定について}
大変形をONにする。

\subsection{解析時の制約}
ひとつのプロジェクトにつき1個しか解析できない。システムをエクスポートで2個目の解析ができる。

しかし、同じプロジェクトを2個開くとバグる。

共用PCはボリュームDに入れる\footnote{QOR(Quality of Research)落ちるよ}。


\newpage


\section{ANSYS生活で今後役に立つかもしれないTips}
\subsection{3Dモデリングソフトの変更}
ANSYSのインストールをミスったとか、周りと逆張りしたいとか、ストライキを起こした(実績あり)とかいう理由でDesign Modeler以外のものにしたいまたはDesign Modelerに戻したい\footnote{要は規定のモデリングソフトを変えたい人(共用PCではDesign Modelerにしとけよ--混乱が巻き起こるぞ)}人に朗報です。再インストールしなくても設定から変更できます。
\begin{enumerate}
  \item 「Workbench」--「ツール」--「オプション」
  \item 「ジオメトリのインポート」
  \item 「使用するジオメトリエディタ」から使用したいエディタを選択
\end{enumerate}


\subsection{3Dモデル作成の簡略化}
インストール時にいわゆる「新基準」でインストールした場合\footnote{川端さんのインストール方法に則っていればインストールされています。}にはAutoCAD・Fusion・SolidWorksのいずれかのCADソフトで作成した3Dモデルをジオメトリとしてインポートできます。

某B4(当時・Sくん)が相当に再インストール時にFusionに対応させるように言っていたので、多分相当便利なのだと思います\footnote{僕みたいに古い人間は分かりません。DesignModelerだけがANSYSで使えるモデラーだと信じてやみません。}。

\date{\today}時点では以下のPCのANSYSは「新基準」でインストールされています。
\begin{itembox}[l]{新基準ANSYS}
  \begin{itemize}
    \item 共用PC
  \end{itemize} 
\end{itembox}

2025/06/25についにこれを書いている本人も時代に取り残されてはいけないと危機感を覚えたようで\footnote{ただの怠惰}、SolidWorksのジオメトリをインポートしてみました。

もちろん、なんと、簡単にできませんでした\footnote{知ってた}。SolidWorksの普通の部品ファイル(SLDPRTファイル)ではインポートができませんでした。対応法としては、非常にめんどくさいのですが、再度SolidWorksを開いて頂いて、保存の際のファイル形式をIGSファイルに変更してください。

内藤さんPCのANSYSではSLDPRTファイルでインポートができていました\footnote{なんでや?}。まぁ、どっかで設定を間違えたのでしょう。


\subsection{圧力単位の勝手に変換}
ANSYSのバグなんだか仕様なんだか知りませんが、ストライキではねえなと思ったのでTipsに。単位系でmm(ミリメートル)を選択すると勝手に圧力の単位がMPa(メガパスカル)になります\footnote{本当になんで?}。つまり、単位系でm(メートル)を選択していれば圧力の単位はPa(パスカル)になるはずです。ANSYSを使用している方で、時刻歴応答解析などで圧力を取り扱う方\footnote{特に音波組}はご注意ください。

圧力の不正入力については、「単位をしっかりと確認する」以外は有効な対応策はない状況です。解析が止まればいいですが、止まらなかったら\underline{クソデカ}圧力を当てて\underline{良さげな}成果を引っ提げて研究報告会に突っ込む可能性があります\footnote{「素人質問で恐縮ですが、音圧高くないですか?」なんて悪夢ですよ}。

なお、圧力の不正入力でエラーが出るのは想定している圧力より\underline{大きい}圧力(想定:Pa(パスカル)・実際:MPa(メガパスカル))をかけてしまい、「大変形エラー」が出るときです。このエラーはなかなか消えませんでしたが、圧力を変えたら一瞬で直りました\footnote{ならそう書け。Dear人間様、圧力が高すぎませんか?って}。

想定より小さい圧力(想定:MPa(メガパスカル)・実際:Pa(パスカル))をかけると、いかにもそれらしい結果を出力します\footnote{出来損ないのAI}。

もっと気をつけてほしいのは、想定よりもさらにさらに小さい圧力(想定:Pa(パスカル)・実際:$\mu$Pa(マイクロパスカル))をかけて\underline{しまった}\footnote{不慮の事故}場合です。こんな結果を胸張って堂々と研究報告会に乗り込んだところ優秀な先輩(Hさん)に鋭い質問をされ、ANSYS解析の闇を暴かれたのはこの私です。

この「圧力」項目について仕様が判明しました。結論としては、MPa(メガパスカル)とPa(パスカル)の変換はしてくれますが、条件があります。たぶん、皆さんの予想通りですが「最初にPa(パスカル)で入力した後に単位系をmm(ミリメートル)に変更」すれば、\underline{換算されます}。が、「最初に圧力の値を入力せずに単位系を変え、単位を確認しない」とすごいことになります\footnote{ご想像にお任せします}。ANSYSなんかが入力したい値なんてわかるはずないですよね\footnote{そらそうだ。XXだもん。}。

前回の解析で単位系をm(メートル)にしたにも関わらず、今回の解析では\underline{少し}気が変わって単位系をmm(ミリメートル)にした人は気をつけてください。脳死でMicrosoft謹製アプリの算出結果をコピペすると優秀な先輩(Hさん)に刺されますよ。


\subsection{mechdbファイルについての考察}
前述の通り、mechdbファイルは文字化けをしていました。しかしながら、engdファイルはxmlファイルの形であったことから、mechdbファイルはengdファイルのコンパイルファイルと考えられます。

また、正常の状態ではSYS(-X)の名前でengdファイルとmechdbファイルがセットになっていることから、コンパイルファイルではなかったとしても、「何か関係があるファイルである」と言えるでしょう。

さらに、それぞれのSYS(-X)の名前のengdファイルとmechdbファイルは同じファイルサイズであることからXの数字に関わらず同じデータが格納されているとも考えられます。このとき、内容データを詳しく精査した訳では無いですが、「異なる内容であった」という通報がありました。

これらのことを勘案すると、SYS(-X).engdとSYS(-X).mechdbはプロジェクト内におけるそれぞれの解析項目の「エンジニアリングデータ」のデータに加えて、解析の種類(モーダル解析・時刻歴応答解析(・静的応答解析))が格納されているのではないかということまではわかってきました。このことから、このような仮説を立てさせていただきます\footnote{理論検討(笑)や実験(笑)などをして論文発表してください。}。

当該事案である、平野さんのプロジェクトでは「エンジニアリングデータ」の項目があるプロジェクトが4つありました(モーダル解析・静的応答解析・時刻歴応答解析1・時刻歴応答解析2)。それに対応するようにSYS(-X)も4つありました(SYS・SYS-1・SYS-2・SYS-3)。


\subsection{マイナーな拡張子を開きたいときの小技(Windows信者用)}
mechdbなどのマイナーな拡張子開きたいけどなーってときありますよね?

言っておきますけど、Windowsってダブルクリックだけがそのファイルの「開き方」ではないですからね\footnote{Windowsの「家電化」への警鐘}。

このようなマイナーな拡張子の中身を確認したいときは「右クリック--メモ帳で開く」を選択してください。結構、こいつ有能です\footnote{なんなら、拡張子無くても開けます。(例:hostsファイル)}。

Macはどうだか知りません。ターミナルならいじれそうな気がするけどね。


\subsection{ANSYSのトラブルシューティングのヒント}
まあ、最悪再インストールすればいいのでしょうけど、極力やりたくないと思いますのでこのトラブルシューティングを覚えてください\footnote{テストに出ます}。

うやうや言う前にエラーメッセージを見てください。エラーメッセージは情報の宝庫ですから\footnote{これはマジ}。C言語の授業でも習ったでしょ。

\begin{itemize}
  \item ライセンスサーバ系→hostsファイルの書き換え
  \item エラーに「メカニカル」という文言→マテリアル系を疑う
  \item エラーに「レイアウト」という文言→レイアウトのリセットを試みる
  \item エラーに「不明なエラー」という文言→考えるのをやめるまたはパソコンを殴る\footnote{絶対にやめて下さい}
\end{itemize}


\subsection{モデルのメッシュの数を簡単に見る方法}
知っている人もいるかも知れませんが、見つけたときに言葉では表せないような快感が脳を突き抜けたので、ここで暴露します。

ちなみにこれを書いている人はこれを知るまでファイルタブ--モデルの情報からいちいちボディを抑制しながら確認していた人です。こんな時代遅れなことやっている人はすぐにこれを実践して下さい。

\begin{itemize}
  \item ジオメトリ--ソリッド--情報
\end{itemize}

たったこれだけ。

\subsection{データベースの破損で開けませんのエラーについて}
事例が2つできたので場合分けで解説。

\begin{itemize}
  \item エラーメッセージのパスの該当mechdbファイルが\underline{ない}場合\par
        おめでとうございます。復旧作業はクソだるいです。このプロジェクトの放棄も含めて検討して下さい\footnote{プロジェクトの新規作成}。

        もし、修正する気力が出たのならば以下の通りに行って下さい。もちろん、自己責任かつ直らなくても書いた人を見つけ出して怒らないでね。
        \begin{enumerate}
          \item エラーメッセージに書かれたパスに行って下さい。
          \item 適当なmechdbファイルをコピーして異常があるファイルのファイル名に変更して下さい。
          \item プロジェクトを開いて下さい。
          \item 整合性を取って下さい(ここが1番めんどくさい)。
          \item 問題がある解析のブロックだけ消して立て直すのもアリです(こっち推奨)。
          \item あとは祈って下さい。
        \end{enumerate}
  \item エラーメッセージのパスの該当mechdbファイルが\underline{ある}場合\par
        非常に残念ながら、復旧作業はそこそこだるいです。しかし、プロジェクトの放棄は早とちりです。やめて下さい。早まらないで。

        多分、修正する気力が湧いていると思うので以下の通り行って下さい。もちろん、自己責任かつ直らなくても書いた人を見つけ出して怒らないでね。
        \begin{enumerate}
          \item エラーメッセージに書かれたパスに行って下さい。
          \item 異常があるmechdbファイルの拡張子をmechdatに変更して下さい。
          \item Workbenchを「新規」で開いて下さい。
          \item mechdatファイルをドラッグアンドドロップして下さい。
          \item 開けるか確認して下さい\footnote{開けなかったら悟って下さい}。
          \item 保存せずに(そっと)閉じて下さい。
          \item 異常が出ていた元のプロジェクトを開いて下さい。
        \end{enumerate}
\end{itemize}
これやってだめなら悟り開いて下さい。ANSYS教でも始めて下さい。


\subsection{補足事項}
\subsubsection{座席相対表}
このファイルを作成していて気がついてしまいました。数年後には全く違う座席配置図になっている可能性があるということを\footnote{倒置法}。

\begin{figure}[ht]
  \centering
  \includegraphics[scale=0.4]{image/lab_seat.png}
\end{figure}

\begin{itembox}[l]{座席相対表(2024/2025)}
  \begin{enumerate}
    \item 麻妻さんPC/麻妻さんPC
    \item 使用者なし/浅見さんPC
    \item 土田さんPC/土田さんPC
    \item 内藤さんPC/内藤さんPC
    \item 使用者なし/使用者なし
    \item 共用PC/共用PC
    \item 澁谷さんPC/森田さんPC
    \item 川端さんPC/速水さんPC
    \item 松岡さんPC/井上さんPC
    \item 成田さんPC/平野さんPC
  \end{enumerate}
\end{itembox}

\newpage


\section{おわりに}
良いANSYSライフを。あんまりANSYSにイライラせずにゆっくり着実に研究を進めましょう。ANSYSが使えなくてもXXX\footnote{DIC}とかXXXX\footnote{概念設計}\footnote{あんまりのめり込むと怒られる}とかやれることはあるので。ANSYSが壊れてるときにしかできないことにもチャレンジしてみましょう。良い成果が出ることをどこからか\footnote{研究室かもしれないし、どっかの会社かもしれないし、ニートだったら河川敷??}応援しております。

もし、この資料を読んでも解決できなかった場合は新種のANSYSのストライキですので至急先生に報告することを強くおすすめいたします。このときの対応は、自分のためにも同僚のためにも後輩のためにも研究室のためにも何かしらの形でメモなりこのようなカス文章なりにまとめておくことをお願いいたします。ANSYSのストライキへの対応は決して「時間の無駄」なんかではなく、「未来への時間の投資」として考えていただけると幸いです。

この文章はANSYSへの怒りを沸々とさせながら書いた文章です。よくわからないところや修正希望、項目の追加希望等ありましたら研究室内で叫ぶとか、ホワイトボードになんかメモっとくとかしてください。それ相応の対応はさせて頂きます。こういう、意味のわからないことをしている人はフィードバックに関しては特に怒ったりはしません。\footnote{ところがどっこい、「こんなの無駄ですよね」とか「なんの役にも立ちませんよね」等の\doublebox{誰の目にもわかるド正論}を言ってきたら、完全無視して怒りまくります(拗ねます)。}どんどんご意見お待ちしております。

末筆ながら皆様の研究の進展を祈念いたしましておわりの言葉とさせて頂きます\footnote{一度は書いてみたかった}。
\quad
\begin{flushright}
  \date{\today}\ \ 材料システム研究室某B4(当時)
\end{flushright}
\end{document}